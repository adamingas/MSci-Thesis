% ************************** Thesis Abstract *****************************
% Use `abstract' as an option in the document class to print only the titlepage and the abstract.
\begin{abstract}
In this project we show that the application of supervised machine learning methods, for the purpose of classification, on metabarcoding eDNA data is feasible. Samples of water were collected and filtered from Amazonian rivers of Peru by WWF using NatureMetric Kits. The eDNA found in the samples were sequenced using high-throughput sequencing and clustered into OTUs using metabarcodes. We use CSS normalisation, several ordination methods, and other dimensionality reduction techniques to create features out of the OTU table produced from sequencing. Several classifiers trained on these feature sets are used to infer the water colour the samples come from. We outline the workings of both Bayesian and Maximum likelihood Logistic Regression, and of Random Forest. Ways to split the data into train-validation-test sets are devised based on the samples' location; the splits are used to evaluate the classifiers' performance under different scenarios. We find that all of them perform much better, conditional on the features set used, than a naive benchmark score that mimics a weighted `coin flip'. It is proposed that extensive sampling efforts, spatially and temporally, together with machine learning methods and time series analysis can be used to uncover complex interactions between the species in a community that otherwise go unnoticed when using classical ecological tools.

% In this project has shown that the application of supervised machine learning methods to eDNA metabarcoding derived data for the classification of species assemblages into water colour is feasible.Furthermore, the classifiers were tested on three splitting methods based on the location of samples, mirroring different prediction conditions.
 
\end{abstract}
%