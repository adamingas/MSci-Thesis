\chapter{Paper Summaries}
%Summaries of various papers that can be used in the literature review part of the thesis
\textit{Predicting the Ecological Quality Status of Marine Environments from eDNA Metabarcoding Data Using Supervised Machine Learning (2017)}
The authors collected sediments from 5 aquatic farms, 24 stations, and 48 grabs. Normally, the assessment of biodiversity involves the identification of benthic macro-invertebrates, which is a time consuming process. The authors used instead High-throughput amplicon sequencing of foraminifera environmental DNA found in the benthic zone and clustered the sequences into Operational Taxonomic Units (OTUs), some of which have not yet been assigned to a morphospecies. Two supervised machine learning models (Random Forrest and Self-Organising Map) were trained to predict the quality of the environment (infer values of Biotic indices), and a correlation screening approach was used to infer the ecological weights associated with each OTU based on morpho-taxonomic inventories. The OTUs were used as inputs in two ways; either their diversity metrics were used, or their composition (number of reads). The machine learning models performed better than the correlation screening one, when compared to BI calculated using morpho-taxonomic inventories.



a summary of different indices commonly used in marine fish farming \cite{BORJA2009231}

Disadvantages of traditional methods that monitor marine ecological status \cite{goodwin_traditional_env_monitoring}



