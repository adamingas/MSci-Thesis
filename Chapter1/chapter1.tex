%!TEX root = ../thesis.tex
%*******************************************************************************
%*********************************** First Chapter *****************************
%*******************************************************************************

\chapter{Background}  %Title of the First Chapter

\ifpdf
    \graphicspath{{Chapter1/Figs/Raster/}{Chapter1/Figs/PDF/}{Chapter1/Figs/}}
\else
    \graphicspath{{Chapter1/Figs/Vector/}{Chapter1/Figs/}}
\fi


%********************************** %First Section  **************************************
%\section{What is loren ipsum? Title with math \texorpdfstring{$\sigma$}{[sigma]}} %Section - 1.1 
\section{Motivation}
Why is measuring the quality of environment important

Legal requirements by eu and other agencies

What do we want to achieve

Environmental studies are conducted all around the world to establish how and in what ways human activities affect their environment, if and what the negative consequences are, and how this change propagates. The sites of interest can be fish farms (for example in New Zealand, Scotland, Norway and others), rivers that cross various landscapes, land (used for dairy-farming, horticulture, or where different kinds of forests grow) and many other sites. A good way to infer the health of an environment is by investigating the species inhabiting it, and in particular their relative abundance; some species are very intolerant of pollution (Alderfly Larva) whereas others are tolerant (Leeches, Blood Worms). Thus, the distribution of the individuals in the various species (or other higher taxonomic groups) can tell us a lot about the levels of pollution. In rivers, the quality of water can be assessed by examining macro invertebrates found in different sites. In land, soil bacterial community composition can be used to infer soil condition and health \cite{hermans_bacteria_2016}.

The health of the environment in a particular area is closely related to finding out their relative abundance might be necessary to evaluate the quality of their habitat. The traditional method of identifying species is morpho-taxonomic, which requires expert knowledge, is time consuming and thus expensive. 

Why is measuring the quality of environment important

Legal requirements by eu and other agencies



Why

\section{Metabarcoding}
Bacteria can be used as bioindicators to determine health of environment. This can be done because bacterial communities respond to changes in soil more than differences in the climate or geography.  bacterial taxonomic groups respond differently to various soil atributes (pH, carbon-to-nitrogen rations, Olsen P\footnote{Measure of plant available phosphorous} etc). Thus, the relative abundance of specific taxa reflect the impact of specific anthropogenic activities on the environment, even when comparing soil samples across large geographical areas.\cite{hermans_bacteria_2016}

benthic bacterial communities react in a
similar way to the same environmental stressors as benthic macrofauna
communities.(Stoeck 2018)
How we can identify bioindicators 




\section{Data}
%Explain topography of river
%Maranon upper is behind a natural barrier that makes it distinct from the other rivers etc
%how data are sampled