%!TEX root = ../thesis.tex
%*******************************************************************************
%*********************************** First Chapter *****************************
%*******************************************************************************

\chapter{Background}  %Title of the First Chapter

\ifpdf
    \graphicspath{{Chapter1/Figs/Raster/}{Chapter1/Figs/PDF/}{Chapter1/Figs/}}
\else
    \graphicspath{{Chapter1/Figs/Vector/}{Chapter1/Figs/}}
\fi


%********************************** %First Section  **************************************
%\section{What is loren ipsum? Title with math \texorpdfstring{$\sigma$}{[sigma]}} %Section - 1.1 
\section{Motivation}

Increased human populations during the industrialisation of the 19th century were accompanied by increasing amounts of unmanaged waste that resulted in public health problems. The first attempts at remedying the issues where applied mostly to running waters, and had a bacteriological focus \cite{AQUATIC_INSECTS_BIOMONITORING}. As time passed, managing freshwater systems became more important and evolved into a more complex procedure that took into account entire aquatic communities (like macro-invertebrates and fish) rather than specific species.



Nearing the end of the 20th century, "ecological health" became a priority in some human societies; people begun pressuring public authorities to restore freshwater systems to a healthier state. This is also reflected in the political sphere with the rise of Green parties in more economically developed nations across the world. Subsequently, huge budgets have been allocated in the management of freshwater, and other, systems; an example being the restoration of the Emscher river system which has an estimated cost of US\$5.5billion \cite{emscher}. 

 The driving motive today for the assessment of environmental consequences (from plans, policies or projects) around the world is coming from regional legislation, operations of Non-Governmental Organisations and requirements set by the financial backers of projects in the area. Legal procedures, policies and instruments are set up to ensure that decision makers take into consideration the environmental impacts of their projects; examples include the Environmental Impact Assessment Directive of the European Union \cite{eia_eu} and the Environmental Protection Agency in the United States \cite{us_epa_our_2013}, both of which became operational around the 1970s.
 
 

Environmental studies conducted all around the world test sites such as fish farms (for example in New Zealand, Scotland, Norway and others) \cite{stoeck_environmental_2018}, rivers that cross various landscapes, land (used for dairy-farming, horticulture, or where different kinds of forests grow)\cite{hermans_bacteria_2016} and many others. A good way to infer the health of an environment is by investigating the species inhabiting it, and in particular their relative abundance\footnote{Species abundance is the number of individuals comprising a species in a particular area. Relative abundance is how individuals in a community are distributed among species.}; some species are very intolerant of pollution (Alderfly Larva) whereas others are tolerant (Leeches, Blood Worms). Thus, the distribution of individuals in the various species (or other higher taxonomic groups) can tell us a lot about the levels of pollution. 

The method of using species as indicators to survey the health of an environment is called biomonitoring. The discipline's aim is to find the ideal bioindicators whose presence or behaviour reflects best a stressor's effect on biota. As an example, in rivers, the quality of water can be assessed by examining macro-invertebrates, fishes \cite{bioindicatorsinrivers}, and bacterial communities \cite{stoeck_environmental_2018} found at different sites. In land, soil bacterial community composition can be used to infer soil condition and health \cite{hermans_bacteria_2016}.

%Explaining current methods of biomonitoring
%morphotaxonomic
The traditional methods of biomonitoring involve a limited, long-scaled site sampling of individual organisms which are then processed and sorted into sample taxonomic units. This process can take months to years, and usually produces data of low information \cite{baird_biomonitoring_2012}. Analysis of ecosystems requires taxonomic expertise across many order and several phyla; species-level identification is hindered by problems arising from co-ordinating the inputs of several experts, and differences in taxonomic refinement.As a result, the identified taxa are often very few in numbers, and are usually the ones deemed as critical (indicator organisms), by experts, for the specific study\cite{cranston_biomonitoring_1990}. 

The resolution of identification (or `taxonomic penetration') stops, more often that not, to taxonomically higher categories (genus, family etc) than species. The reasons for the reduction of penetration are usually not made explicit; most often a more pragmatic approach is taken which seeks to determine individuals to the species level only if the ease of doing it and the time taken permits it \cite{cranston_biomonitoring_1990}. As a result, most of the species which are more difficult to identify are lumped together to larger categories, loosing information in the process. This is especially the case with specious groups of freshwater organisms that occupy the lower levels of the food web, even though they constitute most of the biodiversity in the System and thus have the greatest potential for response to stressors \cite{woodward_biomonitoring_21st}. For example, lumping together species of the Chironomidae genus, because of the difficulty in separating them, reduces the sensitivity of the biomonitoring scheme used \cite{ruse_classification_2010}.

%The health of the environment in a particular area is closely related to finding out their relative abundance might be necessary to evaluate the quality of their habitat. The traditional method of identifying species is morpho-taxonomic, which requires expert knowledge, is time consuming and thus expensive. 


\subsection{Genomics: A new hope?}
The morpho-taxonomic identification of species has been the limiting step in biomonitoring efforts because of the short-supply of taxonomists and prohibitive costs in separating and identifying species. The invention in 1977 of Sanger-based DNA sequencing\footnote{Sequencing is the process by which the order of the Nucleotides (or four bases) in a sample of DNA are found.} which revolutionised all branches of the biological sciences, could not be used for environmental bulk samples because they contained potentially thousands of species, and separating them so as to sequence them was prohibitively difficult. 

However, DNA sequence-based analysis has enabled ecologists to answer questions they would not have been able to without such data. In particular with the advent of DNA barcoding in 2004 \cite{hebert_paul_d._n._biological_2003}, which is technique of identifying species based on short DNA sequences, international efforts have been made to build a taxonomic reference library (Barcode of Life Initiative), and identify unknown speciments to the species-level by comparing their sequence to known ones already catalogued in a reference database \cite{savolainen_vincent_towards_2005}. 
% Despite that, reliance to the methods has been going on for decades despite recent advances in molecular microbiology which aimed at describing assemblages of soil bacteria. 

The emergence of Next-generation sequencing (NGS) platforms produced significant improvements in DNA sequencing technologies. The new platforms can deliver billions of sequence reads per single run, which is orders of magnitude better than traditional Sanger sequencing. As a result there has been a significant drop of sequencing costs per megabases that has been going over the last decades. In particular, the cost of sequencing 1 megabase has gone from \$$10^4$ in 2001, to \$$10^2 - 10^3$ in 2007 and finally to less than \$$10^{-1}$ in 2019 \cite{sequencing_costs}. This, coupled with advancements in DNA- and RNA-based techniques in taxonomic identification \cite{baird_biomonitoring_2012} have made possible the application of metagenetics, the study of genetic material sourced directly from the environment, in ecological studies. 

Normal barcoding standards were designed with the purpose of identifying isolated specimens from intact DNA using Sanger sequencing, so are inapplicable to empirical ecological studies where the samples contain DNA from a mixture of related species  \cite{taberlet_towards_2012}. To solve this problem DNA metabarcoding was developed and made possible by NGS techologies. It is a method for high-throughput multi species identification using degraded DNA found in the environment (eDNA) \cite{taberlet_towards_2012}. The method relies on Barcoding genes which have a rate of mutation such that they are stable within a species but different when compared to others. Examples are the 16S rRNA and the Cytochrome Oxidase 1 \cite{hebert_paul_d._n._biological_2003} genes. 

After the DNA is extracted from an environmental sample, it's barcode region needs to be amplified (multiplied millions of times) before it can be sequenced. This involves selecting the right primers \footnote{Short molecules which provide the starting point for DNA amplification (in other words specify the region to be multiplied)} for the targeted taxonomic groups and using PCR for the amplification. 
%Refers to identification of species assemblages from community DNA using barcode Genes. PCR is carried out with non-specific primers, followed by high-throughput sequencing and bioinformatics processing. Can identify hundreds of species in each sample, and 100+ different samples can be processed in parallel to reduce sequencing cost. 'Read more about metabarcoding by clicking here'

This development allowed scientists to overcome the bottleneck of morpho-taxonomic identification of species. 

The way 
%Metagenomics, the study of genetic material directly sourced from the environment, has the ability to revolutionise biomonitoring and bring a new era of massive ecological data generation.


%%Differences between sangers and NGS
%Explain DNA barcoding and reference libraries

%what was done (taxonomic identification) and what can be done
%Older biomonitoring techniques relied too much on expert opinion and on a limited number of easily identifiable taxa. The methodologies used were not scientific. MEasuring the relative abundance of these taxa is not enough to assess the quality of an environment, especially if it is done in isolation. The whole community has to be taken into account and the way the species interact to get a full picture. \textbf{tocite}{woodward}
% Segway to next generation sequencing 
%Dna sequencing technology has undergone impressive improvements over the last year, with the advent of next generation sequencing. High-throughput sequencing is orders of magnitude cheaper and faster than traditional sanger sequencing. This leap of in sequencing capacity can revolutionise a variety of scientific disciplines \textbf{cite shendure 2008}. One such application of NGS is the health evaluation of habitats. \textbf{cite taberlet 2012}

\section{Metabarcoding}
Bacteria can be used as bioindicators to determine health of environment. This can be done because bacterial communities respond to changes in soil more than differences in the climate or geography.  bacterial taxonomic groups respond differently to various soil atributes (pH, carbon-to-nitrogen rations, Olsen P\footnote{Measure of plant available phosphorous} etc). Thus, the relative abundance of specific taxa reflect the impact of specific anthropogenic activities on the environment, even when comparing soil samples across large geographical areas.\cite{hermans_bacteria_2016}

benthic bacterial communities react in a
similar way to the same environmental stressors as benthic macrofauna
communities.(Stoeck 2018)
How we can identify bioindicators 


%6229 have to reach 6729

\section{Data}
\subsection{Geography}
%Explain topography of river
%Maranon upper is behind a natural barrier that makes it distinct from the other rivers etc
\subsection{Sampling}
%how data are sampled
\subsection{Preparation}
%PCR
%Next Generation sequencing
%Biases that come with sampling
%Removal of unidentified species
